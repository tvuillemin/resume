%% If you need to pass whatever options to xcolor
\PassOptionsToPackage{dvipsnames}{xcolor}
\documentclass[10pt,a4paper]{altacv}
\geometry{left=1cm,
          right=9cm,
          marginparwidth=6.8cm,
          marginparsep=1.2cm,
          top=1.25cm,
          bottom=1.25cm}

\usepackage[utf8]{inputenc}
\usepackage[T1]{fontenc}
\usepackage[default]{lato}
\usepackage{eurosym}

\definecolor{Mulberry}{HTML}{72243D}
\definecolor{SlateGrey}{HTML}{2E2E2E}
\definecolor{LightGrey}{HTML}{666666}
\colorlet{heading}{Sepia}
\colorlet{accent}{Mulberry}
\colorlet{emphasis}{SlateGrey}
\colorlet{body}{LightGrey}

% Change the bullets for itemize and rating marker
% for \cvskill if you want to
\renewcommand{\itemmarker}{{\small\textbullet}}
\renewcommand{\ratingmarker}{\faCircle}

%% sample.bib contains your publications
% \addbibresource{sample.bib}

\begin{document}
\name{Thibaut Vuillemin}
\tagline{Ingénieur recherche et développement en informatique}
\photo{2.8cm}{Avatar}
\personalinfo{%
  % Not all of these are required!
  % You can add your own with \printinfo{symbol}{detail}
  \email{contact@thibautvuillemin.com}
  \phone{06 42 00 42 35}
  %\mailaddress{Address, Street, 00000 County}
  \location{Lyon, France}
  %\homepage{www.thibautvuillemin.com}
  %\twitter{@twitterhandle}
  \printinfo{\faMale}{31 ans}
  \linebreak
  \linkedin{linkedin.com/in/thibautvuillemin/}
  \github{github.com/tvuillemin}
  \printinfo{\faCar}{Véhiculé}
  %% You MUST add the academicons option to \documentclass, then compile with LuaLaTeX or XeLaTeX, if you want to use \orcid or other academicons commands.
%   \orcid{orcid.org/0000-0000-0000-0000}
}

%% Make the header extend all the way to the right, if you want. Extend the right margin by 8cm (=6.8cm marginparwidth + 1.2cm marginparsep)
\begin{adjustwidth}{}{-8cm}
\makecvheader
\end{adjustwidth}

%% Provide the file name containing the sidebar contents as an optional parameter to \cvsection.
%% You can always just use \marginpar{...} if you do
%% not need to align the top of the contents to any
%% \cvsection title in the "main" bar.
\cvsection[sidebar]{Postes}

\cvevent{Ingénieur R\&D}{LumApps}{Depuis Janvier 2018}{Lyon, France}
\begin{itemize}
\item Développement de services backend, pour web et mobile
\item R\&D sur la mise en pratique de la \textit{Clean Architecture}
\item Migration de données en masse sur une nouvelle infrastructure
\item Déploiement sur Microsoft Azure et Google Cloud Platform 
\item Mise en place d'intégration et de déploiement continus
\end{itemize}

\divider

\cvevent{Ingénieur R\&D}{Stormshield}{Septembre 2015 -- Janvier 2018}{Lyon, France}
\begin{itemize}
\item Développement d'un service cloud de détection de malwares
\item Développement d'un portail web pour explorer la base de malwares
\item Gestion et amélioration des services cloud de l'entreprise
%\item Développement d'un SDK pour firewalls
\end{itemize}

\divider

\cvevent{Ingénieur logiciel}{Kien}{Octobre 2014 -- Juillet 2015}{Delft, Pays-Bas}
%\begin{itemize}
%\item Prototypage d'enceintes sonores sans fils et multi-usages
%\end{itemize}

\divider

\cvevent{Ingénieur validation}{Alstom Transport {\footnotesize via Sogeti HT}}{Octobre 2012 -- Octobre 2014}{Lyon, France}
%\begin{itemize}
%\item Validation de systèmes informatiques embarqués sur trains
%\end{itemize}

%\divider

%\cvevent{Stage}{ARM Ltd.}{Juin 2011 -- Octobre 2011}{Cambridge, Royaume-Uni}

%\cvevent{Pendant les études}
%\begin{itemize}
%\item Stage de 4 mois chez ARM à Cambridge, UK
%\end{itemize}


%\cvsection{Projects}
%
%\cvevent{Project 1}{Funding agency/institution}{}{}
%\begin{itemize}
%\item Details
%\end{itemize}
%
%\divider
%
%\cvevent{Project 2}{Funding agency/institution}{Project duration}{}
%A short abstract would also work.
%
%\medskip

\cvsection{Technique}

% Adapted from @Jake's answer from http://tex.stackexchange.com/a/82729/226
% \wheelchart{outer radius}{inner radius}{
% comma-separated list of value/text width/color/detail}
\cvachievement{\faCode}{Programmation}{Langages bas-niveau et langages de script}
\wheelchart{1.5cm}{0.5cm}{%
  4/8em/accent!25/Bash,
  8/8em/accent!75/Python,
  2/8em/accent!25/Golang,
  4/8em/accent!50/C,
  3/8em/accent!25/Javascript,
  1/8em/accent!75/Rust
}
\bigskip

\cvachievement{\faTerminal}{Systèmes}{CfM, Conteneurs, et Bases de données}
\wheelchart{1.5cm}{0.5cm}{%
  6/8em/accent!75/PostgreSQL,
  2/8em/accent!25/Ansible,
  4/8em/accent!50/Docker,
  4/8em/accent!75/Kubernetes,
  2/8em/accent!25/Terraform,
  4/8em/accent!50/MongoDB
}
%\bigskip

%\cvachievement{\faUsers}{Human skills}{Interpersonal and soft skills}
%\cvskill{Teamwork}{5}
%\cvskill{Critical thinking}{4}
%\cvskill{Public speaking}{3}
%\cvskill{Conflict resolution}{2}

%\clearpage
%\cvsection[page2sidebar]{Publications}
%
%\nocite{*}
%
%\printbibliography[heading=pubtype,title={\printinfo{\faBook}{Books}},type=book]
%
%\divider
%
%\printbibliography[heading=pubtype,title={\printinfo{\faFileTextO}{Journal Articles}},type=article]
%
%\divider
%
%\printbibliography[heading=pubtype,title={\printinfo{\faGroup}{Conference Proceedings}},type=inproceedings]

%% If the NEXT page doesn't start with a \cvsection but you'd
%% still like to add a sidebar, then use this command on THIS
%% page to add it. The optional argument lets you pull up the 
%% sidebar a bit so that it looks aligned with the top of the
%% main column.
% \addnextpagesidebar[-1ex]{page3sidebar}


\end{document}
